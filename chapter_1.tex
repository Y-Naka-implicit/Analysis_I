
\section{数}

\subsection{実数}
\subsection{自然数,整数}
\subsection{順序体}
\subsubsection{}
\subsubsection{}
\begin{ans}
    $L(A)$を$A$の下界,$U(-A)$を$-A$の上界とする:
    \begin{align*}
        & L(A) = \{ l \in \mathbb{R} ~ | ~ \forall a \in A ~ l \leq a \}\\
        & U(-A) = \{ -u \in \mathbb{R} ~ | ~ \forall -a \in -A ~ -a \leq -u \}.
    \end{align*}
    いま
    \begin{align*}
        & -m = \sup (-A)\\
        \Leftrightarrow & \begin{cases}
            -m \in U(-A)\\
            \forall -u \in U(-A) ~ -m \leq -u
        \end{cases}\\
        \Leftrightarrow & \begin{cases}
            m \in -U(-A)\\
            \forall u \in -U(-A) ~ u \leq m.
        \end{cases}
    \end{align*}
    もし$-U(-A) = L(A)$なら,
    \begin{align*}
        \Leftrightarrow & \begin{cases}
            m \in L(A)\\
            \forall u \in L(A) ~ u \leq m
        \end{cases}\\
        \Leftrightarrow & m = \inf A
    \end{align*}
    である.以下,$-U(-A) = L(A)$を示す.
\end{ans}


\subsection{実数体の構成}
\subsection{複素数}